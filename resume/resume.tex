\documentclass[10pt, letterpaper]{article}

% Packages:
\usepackage[
    ignoreheadfoot, % set margins without considering header and footer
    top=2 cm, % seperation between body and page edge from the top
    bottom=2 cm, % seperation between body and page edge from the bottom
    left=2 cm, % seperation between body and page edge from the left
    right=2 cm, % seperation between body and page edge from the right
    footskip=1.0 cm, % seperation between body and footer
    % showframe % for debugging
]{geometry} % for adjusting page geometry
\usepackage{titlesec} % for customizing section titles
\usepackage{tabularx} % for making tables with fixed width columns
\usepackage{array} % tabularx requires this
\usepackage[dvipsnames]{xcolor} % for coloring text
\definecolor{primaryColor}{RGB}{0, 0, 0} % define primary color
\usepackage{enumitem} % for customizing lists
\usepackage{fontawesome5} % for using icons
\usepackage{amsmath} % for math
\usepackage[
    pdftitle={Christopher Odom's Resume},
    pdfauthor={Christopher Odom},
    colorlinks=true,
    urlcolor=primaryColor
]{hyperref} % for links, metadata and bookmarks
\usepackage[pscoord]{eso-pic} % for floating text on the page
\usepackage{calc} % for calculating lengths
\usepackage{bookmark} % for bookmarks
\usepackage{lastpage} % for getting the total number of pages
\usepackage{changepage} % for one column entries (adjustwidth environment)
\usepackage{paracol} % for two and three column entries
\usepackage{ifthen} % for conditional statements
\usepackage{needspace} % for avoiding page brake right after the section title
\usepackage{iftex} % check if engine is pdflatex, xetex or luatex

% Ensure that generate pdf is machine readable/ATS parsable:
\ifPDFTeX
    \input{glyphtounicode}
    \pdfgentounicode=1
    \usepackage[T1]{fontenc}
    \usepackage[utf8]{inputenc}
    \usepackage{lmodern}
\fi

\usepackage{charter}

% Some settings:
\raggedright
\AtBeginEnvironment{adjustwidth}{\partopsep0pt} % remove space before adjustwidth environment
\pagestyle{empty} % no header or footer
\setcounter{secnumdepth}{0} % no section numbering
\setlength{\parindent}{0pt} % no indentation
\setlength{\topskip}{0pt} % no top skip
\setlength{\columnsep}{0.15cm} % set column seperation
\pagenumbering{gobble} % no page numbering

\titleformat{\section}{\needspace{4\baselineskip}\bfseries\large}{}{0pt}{}[\vspace{1pt}\titlerule]

\titlespacing{\section}{
    % left space:
    -1pt
}{
    % top space:
    0.3 cm
}{
    % bottom space:
    0.2 cm
} % section title spacing

\renewcommand\labelitemi{$\vcenter{\hbox{\small$\bullet$}}$} % custom bullet points
\newenvironment{highlights}{
    \begin{itemize}[
        topsep=0.10 cm,
        parsep=0.10 cm,
        partopsep=0pt,
        itemsep=0pt,
        leftmargin=0 cm + 10pt
    ]
}{
    \end{itemize}
} % new environment for highlights


\newenvironment{highlightsforbulletentries}{
    \begin{itemize}[
        topsep=0.10 cm,
        parsep=0.10 cm,
        partopsep=0pt,
        itemsep=0pt,
        leftmargin=10pt
    ]
}{
    \end{itemize}
} % new environment for highlights for bullet entries

\newenvironment{onecolentry}{
    \begin{adjustwidth}{
        0 cm + 0.00001 cm
    }{
        0 cm + 0.00001 cm
    }
}{
    \end{adjustwidth}
} % new environment for one column entries

\newenvironment{twocolentry}[2][]{
    \onecolentry
    \def\secondColumn{#2}
    \setcolumnwidth{\fill, 6.0 cm}
    \begin{paracol}{2}
}{
    \switchcolumn \raggedleft \secondColumn
    \end{paracol}
    \endonecolentry
} % new environment for two column entries

\newenvironment{header}{
    \setlength{\topsep}{0pt}\par\kern\topsep\centering\linespread{1.5}
}{
    \par\kern\topsep
} % new environment for the header

% save the original href command in a new command:
\let\hrefWithoutArrow\href

% new command for external links:

\begin{document}
    \newcommand{\AND}{\unskip
        \cleaders\copy\ANDbox\hskip\wd\ANDbox
        \ignorespaces
    }
    \newsavebox\ANDbox
    \sbox\ANDbox{$|$}

    \begin{header}
        \fontsize{25 pt}{25 pt}\selectfont Christopher Odom

        \vspace{5 pt}

        \normalsize
        \mbox{Greater Boston, MA}%
        \kern 5.0 pt%
        \AND%
        \kern 5.0 pt%
        \mbox{\hrefWithoutArrow{mailto:christopher.r.odom@gmail.com}{christopher.r.odom@gmail.com}}%
        \kern 5.0 pt%
        \AND%
        \kern 5.0 pt%
        \mbox{\hrefWithoutArrow{tel:+781-475-3804}{781 475 3804}}%
        \kern 5.0 pt%
        \AND%
        \kern 5.0 pt%
        \mbox{\hrefWithoutArrow{https://codom.github.io/}{codom.github.io}}%
        \kern 5.0 pt%
        \AND%
        \kern 5.0 pt%
        \mbox{\hrefWithoutArrow{https://linkedin.com/in/christopher-r-odom}{linkedin.com/in/christopher-r-odom}}%
        \kern 5.0 pt%
        \AND%
        \kern 5.0 pt%
        \mbox{\hrefWithoutArrow{https://github.com/codom}{github.com/codom}}%
    \end{header}

    \vspace{5 pt - 0.3 cm}

    % \section{Summary}
    %     \begin{onecolentry}
    %         \textit{Embedded Systems Engineer with over 5 years of experience in low-level development, custom Linux builds, and driver development. Seeking contracting opportunities to leverage expertise in ARM-based platforms, RTOS, and full-stack embedded solutions.}
    %     \end{onecolentry}

    \section{Experience}
        \begin{twocolentry}{
            Nov 2024 – Present
        }
            \textbf{Cofounder \& Lead Embedded Engineer}, Stealth Startup -- Greater Boston, MA
        \end{twocolentry}

        \vspace{0.10 cm}
        \begin{onecolentry}
            \begin{highlights}
                \item Spearheaded the development of a custom embedded Linux distribution using the Yocto Project for ARM Cortex-based platforms.
                \item Authored and integrated custom drivers for I2C and SPI-based sensors, enabling core product functionality.
                \item Engineered and deployed applications on diverse embedded targets, including bare-metal/RTOS (Zephyr on STM32, ESP32) and embedded Linux systems.
                \item Implemented display logic for a MIPI DSI interface, contributing to the device's user interface.
            \end{highlights}
        \end{onecolentry}
        
        \vspace{0.2 cm}

        \begin{twocolentry}{
            November 2023 – Present
        }
            \textbf{Data Labeler for Software}, Data Annotation -- Remote, US
        \end{twocolentry}

        \vspace{0.10 cm}
        \begin{onecolentry}
            \begin{highlights}
                \item Labelled datasets for coding and data-science workflows.
                \item Created various workflows to optimize deploying projects in React, Android, Python, and C++.
                \item Leveraged AI to build various file-hosting, music streaming, and blogging apps.
            \end{highlights}
        \end{onecolentry}

        \vspace{0.2 cm}

        \begin{twocolentry}{
            June 2018 – Jan 2019
        }
            \textbf{Software Engineering Intern}, Draper -- Cambridge, MA
        \end{twocolentry}

        \vspace{0.10 cm}
        \begin{onecolentry}
            \begin{highlights}
              \item Designed and implemented a test-application framework for hardware based compliance testing.
              \item Built custom ioctls to integrate special hardware tests to an offline data collection pipeline.
              \item Created custom internal tooling to collect data and configure specialized tests.
            \end{highlights}
        \end{onecolentry}

    \section{Projects}
        \begin{twocolentry}{
            \href{https://github.com/codom/codom.github.io}{github.com/codom/codom.github.io}
        }
        \textbf{Personal Website}
        \end{twocolentry}

        \vspace{0.10 cm}
        \begin{onecolentry}
            \begin{highlights}
              \item Personal landing page for experimenting with web development (Vue.js, three.js, GLSL).
            \end{highlights}
        \end{onecolentry}

        \vspace{0.2 cm}

        \begin{twocolentry}{
            \href{https://github.com/Codom/SimpleGuitarAmp}{github.com/Codom/SimpleGuitarAmp}
        }
            \textbf{Guitar Amp Sim}
        \end{twocolentry}

        \vspace{0.10 cm}
        \begin{onecolentry}
            \begin{highlights}
                \item Implemented various filter algorithms in Zig to simulate a guitar amp for digital audio workstations.
            \end{highlights}
        \end{onecolentry}

    \section{Additional Experience}
        \begin{onecolentry}
            \textbf{2D Game Engine (2023-2024):} Developed a 2D game engine in Zig on top of Raylib, featuring a custom scripting language, async asset management, and interactive event systems.
        \end{onecolentry}

        \vspace{0.2 cm}

        \begin{onecolentry}
            \textbf{Caretaker (2021-2025):} Assisted family with end-of-life planning and care.
        \end{onecolentry}
        
        \vspace{0.2 cm}

        \begin{onecolentry}
            \textbf{CCDC (2020):} Participated in a competitive regional cybersecurity competition.
        \end{onecolentry}

    \section{Skills}
        \begin{onecolentry}
            \textbf{Embedded Systems:} Yocto Project, Zephyr RTOS, Bare-metal, I2C, SPI, MIPI DSI, ARM Cortex-M \\
            \textbf{Platforms:} STM32, ESP32, Embedded Linux \\
            \textbf{Languages:} C, C++, Zig, Python, JavaScript, SQL \\
            \textbf{Tools \& Software:} Git, Docker, Podman, LLVM, Linux, GDB
        \end{onecolentry}

    \section{Education}
        \begin{twocolentry}{
            Sept 2017 – May 2021
        }
            \textbf{University of Massachusetts, Lowell}, BS in Computer Science
        \end{twocolentry}

        \vspace{0.10 cm}
        \begin{onecolentry}
            \begin{highlights}
                \item \textbf{Relevant Coursework:} Computer Architecture, Operating Systems, Analysis of Algorithms
            \end{highlights}
        \end{onecolentry}
\end{document}
